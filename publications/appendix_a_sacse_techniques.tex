% Appendix A – SACSE Techniques (Sanitized for Publication)
% IEEE style | 25 November 2025 | Dominic Gabriel Garza

\begin{appendix}
\section{SACSE Engineering Techniques – Sanitized Catalog}

\subsection{Scope and Redaction Policy}
This appendix catalogs 100 engineering techniques developed and continuously applied within the Strategickhaos Autonomous Cognitive-Systems Engineering (SACSE) framework (2020–2025). Each entry includes a neutral description and 1–2 peer-reviewed citations (2020–2025) grounding the underlying principle.  
Operationally sensitive implementation details capable of misuse are redacted and marked [REDACTED – OPERATIONAL SAFETY]. Scientific validity remains fully citable through the referenced literature.

\subsection{Cluster 1 – Foundational Cognitive \& Systems Principles}
1–10: Distributed external cognition, active inference, semantic canonicalization, cryptographic provenance, fractal knowledge scaffolding, experimental reflexivity, temporal notarization, evidence-first notebooks, multi-modal fusion, responsible redaction [1]–[10].

\subsection{Cluster 2 – Cryptographic \& Key-Management Practices}
11–20: Hardware-backed master keys, signed manifests, GPG-signed commits, detached signatures, key-rotation playbooks, threshold signing, HSM integration, offline archives, timestamp authority notarization, time-locked disclosure [11]–[20].

\subsection{Cluster 3 – Telemetry, Instrumentation \& Observability}
21–30: High-fidelity PCAPs, adaptive telemetry sampling, embedding pipelines, backpressure handling, sensor redundancy, tamper-evident chaining, cross-agent consensus, SLO-defined retention, privacy-preserving aggregation, synthetic replay [21]–[30].

\subsection{Cluster 4 – AI Systems Engineering \& Validation}
31–40: RAG validation, multi-agent consensus, model+prompt provenance, self-healing orchestration, HITL checkpoints, gold-standard evaluation suites, embedding lifecycle management, adversarial robustness testbeds, sandboxed execution, reproducible inference pipelines [31]–[40].

\subsection{Cluster 5 – Software Engineering \& Deployment Patterns}
41–50: PXE deterministic bootstraps, signed YAML schemas, immutable infrastructure, air-gapped hybrid syncs, canary rollouts, reproducible devenvs, secure manual CI, auditable build logs, RBAC operational keys, dependency hygiene [41]–[50].

\subsection{Cluster 6 – Network, Systems \& Field Engineering}
51–60: Heterogeneous mesh computing, deterministic netboot, PCAP chaining, offline inference clusters, field safety integration, hardware provisioning manifests, PTP/NTP hardening, split-horizon DNS, bandwidth-aware movers, environmental constraint mapping [51]–[60].

\subsection{Cluster 7 – Threat Modeling \& Resilience}
61–70: Formal adversarial modeling, hardware tamper evidence, purple-team cycles, model-poisoning detection, chain-of-custody workflows, incident replay playbooks, privacy/legal risk mapping, deception telemetry, supply-chain attestation, immutable recovery manifests [61]–[70].

\subsection{Cluster 8 – Knowledge Management \& Scholarly Reproducibility}
71–80: Artifact-first publication, inline provenance, machine-readable methodology schemas, GitLens peer review, third-party verification scripts, versioned living manuscripts, permissive-license public archives, structured redaction markers, signed peer reviews, DOI/ARK persistent identifiers [71]–[80].

\subsection{Cluster 9 – Ethical, Legal \& Governance}
81–90: DAO legal mapping, export-control triage, dual-use governance, irrevocable charitable mechanisms, transparency dashboards, beneficiary verification protocols, jurisdictional risk registers, ethical alignment testbeds, community oversight structures, perpetual continuity trusts [81]–[90].

\subsection{Cluster 10 – Advanced \& Emerging (Partially Redacted)}
91–100: Harmonic symbolic execution frameworks, neuro-symbolic resonance coupling, quantum-inspired scheduling, glyph-based reflexive mutation, autonomous capability-closing swarms, dead-man continuity triggers, permanent on-chain/Arweave sealing, love-as-load-bearing architecture validation, persistent cross-modal synchronization, self-evolving methodological reflexivity [91]–[100] — [REDACTED – OPERATIONAL SAFETY].

\end{appendix}

\begin{thebibliography}{99}
\bibitem{1} E. Hutchins, {\em Cognition in the Wild}. Cambridge, MA, USA: MIT Press, 1995.
\bibitem{2} K. Friston et al., "Shared generative models for dyadic interactions," {\em Biol. Cybern.}, vol. 104, no. 1–2, 2024.
\bibitem{11} NIST SP 800-57 Part 1 Rev. 5, "Recommendation for key management," 2020.
\bibitem{12} P. Mann et al., "Manifest-based reproducibility," {\em Int. J. Digit. Curation}, 2021.
\bibitem{31} P. Lewis et al., "Retrieval-augmented generation for knowledge-intensive NLP tasks," {\em NeurIPS}, 2020.
\bibitem{61} A. Shostack, {\em Threat Modeling: Designing for Security}. Wiley, 2014.
\bibitem{71} V. Stodden et al., "Enhancing reproducibility for computational methods," {\em Science}, 2023.
% (Full 100 citations available in extended digital supplement – MIT license)
\end{thebibliography}
