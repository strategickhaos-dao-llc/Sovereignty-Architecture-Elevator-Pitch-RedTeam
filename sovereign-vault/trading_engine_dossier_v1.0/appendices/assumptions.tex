%%%%%%%%%%%%%%%%%%%%%%%%%%%%%%%%%%%%%%%%%%%%%%%%%%%%%%%%%%%%%%%%%%%%%%%%%%%%%%%%
%% Mathematical Assumptions
%% Strategickhaos Trading Engine v1.0
%% Patent Pending - CONFIDENTIAL
%%%%%%%%%%%%%%%%%%%%%%%%%%%%%%%%%%%%%%%%%%%%%%%%%%%%%%%%%%%%%%%%%%%%%%%%%%%%%%%%

\section*{Appendix C: Mathematical Assumptions and Conditions}

This appendix formally states and justifies the mathematical assumptions underlying the PID-RANCO controller convergence analysis and the broader system design.

\subsection*{C.1 Market Assumptions}

\begin{assumption}[A1: Bounded Markets]
Asset prices remain within finite bounds for all time:
\begin{equation}
\forall i \in \{1, \ldots, n\}, \forall t \geq 0: \quad p_{min}^{(i)} \leq P_i(t) \leq p_{max}^{(i)}
\end{equation}
where $P_i(t)$ is the price of asset $i$ at time $t$, and $0 < p_{min}^{(i)} < p_{max}^{(i)} < \infty$.
\end{assumption}

\textbf{Justification}: This assumption excludes infinite price movements (which would violate limited liability of assets) and negative prices (which are economically meaningless for equities and most crypto assets). Note that this does not preclude arbitrarily large but finite price movements.

\textbf{Practical Note}: The system includes circuit breakers that pause trading if prices approach boundary regions, ensuring this assumption holds during operation.

\begin{assumption}[A2: Continuous Trading]
The market is open and sufficiently liquid during all operation hours, allowing execution of computed trades within bounded slippage:
\begin{equation}
|P_{exec} - P_{quoted}| \leq \Delta_{max}
\end{equation}
where $P_{exec}$ is the execution price, $P_{quoted}$ is the quoted price at order submission, and $\Delta_{max}$ is the maximum acceptable slippage.
\end{assumption}

\textbf{Justification}: This assumption is standard in algorithmic trading literature. It excludes market closures, extreme illiquidity events, and exchange outages from the convergence analysis.

\textbf{Practical Note}: The Low Liquidity (ρ=4) and Crisis (ρ=5) regimes trigger conservative strategies that account for partial violations of this assumption.

\begin{assumption}[A3: Bounded Volatility]
Realized volatility is bounded during operation:
\begin{equation}
\sigma(t) = \sqrt{\frac{1}{W} \sum_{i=t-W}^{t} (r_i - \bar{r})^2} < \sigma_{max}
\end{equation}
where $r_i$ are returns, $W$ is the estimation window, and $\sigma_{max}$ is a finite upper bound.
\end{assumption}

\textbf{Justification}: Unbounded volatility would imply infinite price movements in finite time, violating A1. Empirically, even extreme events like flash crashes have bounded volatility over meaningful time windows.

\textbf{Parameter Selection}: We set $\sigma_{max} = 5\sigma_{historical}$ based on analysis of 2008-2022 market data, which captures all observed volatility spikes including COVID-19 and crypto market crashes.

\subsection*{C.2 Controller Assumptions}

\begin{assumption}[A4: Non-Degenerate Gains]
The proportional gain matrix $K_p$ is positive definite:
\begin{equation}
K_p \succ 0 \quad \Leftrightarrow \quad \lambda_{min}(K_p) > 0
\end{equation}
\end{assumption}

\textbf{Justification}: Positive definiteness ensures that the controller responds to errors in all portfolio dimensions and that the feedback is stabilizing rather than destabilizing.

\textbf{Practical Note}: The adaptive gain scheduling (Section 2.4) maintains positive definiteness by construction for all regime configurations.

\begin{assumption}[A5: Gain Matrix Compatibility]
The derivative gain $K_d$ satisfies:
\begin{equation}
\det(I + K_d) \neq 0
\end{equation}
ensuring the closed-loop system is well-defined.
\end{assumption}

\textbf{Justification}: This is a mild assumption ensuring that the derivative term does not create algebraic loops in the controller.

\textbf{Verification}: For all regime-specific gain matrices, we verify this condition offline during gain tuning.

\begin{assumption}[A6: Target Allocation Boundedness]
The XAI-generated target allocations satisfy feasibility constraints:
\begin{equation}
\mathbf{p}^*(t) \in \mathcal{F} = \left\{\mathbf{p} : \mathbf{1}^T \mathbf{p} = 1, \mathbf{p} \geq 0, \mathbf{p} \leq \mathbf{u}_{max}\right\}
\end{equation}
for all $t$.
\end{assumption}

\textbf{Justification}: The XAI layer is designed to only output feasible allocations, ensuring the RANCO optimization has a non-empty feasible set.

\subsection*{C.3 Stochastic Assumptions}

\begin{assumption}[A7: Sub-Gaussian Returns]
Asset returns are sub-Gaussian with parameter $\sigma^2$:
\begin{equation}
\mathbb{E}[\exp(\lambda (r - \mathbb{E}[r]))] \leq \exp\left(\frac{\lambda^2 \sigma^2}{2}\right) \quad \forall \lambda \in \mathbb{R}
\end{equation}
\end{assumption}

\textbf{Justification}: Sub-Gaussianity is weaker than Gaussianity but sufficient for concentration inequalities. Empirical evidence suggests most asset returns satisfy this condition with appropriate $\sigma^2$.

\textbf{Application}: This assumption enables the probabilistic convergence bound ($1 - \delta$) in Theorem 1.

\begin{assumption}[A8: Mixing Condition]
The market regime process $\{\rho_t\}$ is $\beta$-mixing with exponentially decaying mixing coefficients:
\begin{equation}
\beta(k) \leq C \cdot \exp(-\gamma k) \quad \text{for constants } C, \gamma > 0
\end{equation}
\end{assumption}

\textbf{Justification}: This ensures that distant observations become approximately independent, enabling consistent regime estimation via the HMM.

\subsection*{C.4 Computational Assumptions}

\begin{assumption}[A9: Numerical Precision]
All computations are performed with sufficient numerical precision such that:
\begin{equation}
\|\mathbf{x}_{computed} - \mathbf{x}_{exact}\| \leq \epsilon_{machine}
\end{equation}
where $\epsilon_{machine} \approx 10^{-15}$ for 64-bit floating point.
\end{assumption}

\textbf{Practical Note}: We use double precision (float64) throughout and include numerical stability safeguards (e.g., $\epsilon$ terms in division).

\begin{assumption}[A10: Execution Latency]
Trade execution occurs within the sampling period:
\begin{equation}
t_{execute} < \Delta t
\end{equation}
where $\Delta t$ is the controller sampling period.
\end{assumption}

\textbf{Justification}: This ensures the discrete-time model accurately represents the system behavior.

\subsection*{C.5 Summary Table}

\begin{table}[h]
\centering
\caption{Assumption Summary and Violation Consequences}
\begin{tabular}{clll}
\toprule
\textbf{ID} & \textbf{Assumption} & \textbf{If Violated} & \textbf{Mitigation} \\
\midrule
A1 & Bounded Markets & Price blowup & Circuit breakers \\
A2 & Continuous Trading & Execution failure & Regime detection \\
A3 & Bounded Volatility & Convergence fails & Conservative gains \\
A4 & Non-Degenerate Gains & Unstable control & Offline verification \\
A5 & Gain Compatibility & Undefined system & Design constraint \\
A6 & Feasible Targets & RANCO infeasible & XAI projection \\
A7 & Sub-Gaussian Returns & Prob. bound fails & Robust estimators \\
A8 & Mixing Condition & Regime inconsistent & Larger windows \\
A9 & Numerical Precision & Round-off errors & Double precision \\
A10 & Execution Latency & Model mismatch & Shorter intervals \\
\bottomrule
\end{tabular}
\end{table}

\subsection*{C.6 Relaxations and Extensions}

The following relaxations are possible with modified analysis:

\begin{enumerate}
\item \textbf{Heavy-Tailed Returns}: Replace sub-Gaussian (A7) with sub-exponential or polynomial tail bounds. Convergence rates degrade but stability preserved.

\item \textbf{Model Uncertainty}: Add robust control terms to handle uncertainty in regime parameters. Requires $H_\infty$ analysis.

\item \textbf{Asynchronous Updates}: Allow variable sampling periods by treating $\Delta t$ as time-varying. Requires switched systems analysis.

\item \textbf{Partial Observability}: Extend to handle delayed or noisy state observations using Kalman filtering.
\end{enumerate}
