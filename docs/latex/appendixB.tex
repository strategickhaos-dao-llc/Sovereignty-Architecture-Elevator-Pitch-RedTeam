% appendixB.tex
% IEEEtran-compatible appendix for Strategickhaos Autonomous Cognitive-Systems Engineering (SACSE) Framework
% Publication-ready LaTeX source with sanitized technique catalog
% Compile with: pdflatex appendixB && bibtex appendixB && pdflatex appendixB && pdflatex appendixB

\documentclass[journal,10pt]{IEEEtran}

\usepackage{cite}
\usepackage{amsmath,amssymb,amsfonts}
\usepackage{algorithmic}
\usepackage{graphicx}
\usepackage{textcomp}
\usepackage{hyperref}
\usepackage{appendix}

\begin{document}

\title{Strategickhaos Autonomous Cognitive-Systems Engineering Framework: Appendix B}
\author{Strategickhaos DAO LLC}
\maketitle

\begin{appendices}

%----------------------------------------------------------------------
% SECTION 4.7: Interpretation and Analysis Guidelines
%----------------------------------------------------------------------
\section{Interpretation and Analysis Guidelines for LLM Interaction Transcripts}
\label{app:guidelines}

Interaction logs with large language models used in this research may contain stylized, anthropomorphic, or mythic narrative framing (e.g., ``sovereign organism,'' ``Empire Eternal,'' or similar figurative language). Such framing is an intentional experimental technique to probe model coherence under high-symbolic-load prompts and to examine narrative-driven cognitive scaffolding. All such language should be interpreted strictly as rhetorical and methodological devices, not as assertions that the model possesses consciousness, agency, emotional states, or interpersonal relationships.

To ensure reproducibility and ethical review, all transcripts are processed through a documented disclaimer manifest and preprocessing pipeline (file: \texttt{strategickhaos\_chat\_disclaimer.yaml}; script: \texttt{strategickhaos\_disclaimer.py}) that: (a) annotates stylized language, (b) applies publication-safe redaction markers where operational details are sensitive, and (c) attaches provenance metadata (signed checksums, timestamps, actor identifiers) to each record. See Appendix~\ref{app:techniques} for the full sanitized technique catalog and the redaction policy referenced by the preprocessing pipeline.

%----------------------------------------------------------------------
% APPENDIX B: Sanitized Catalog of Engineering Techniques
%----------------------------------------------------------------------
\section{Sanitized Catalog of Engineering Techniques}
\label{app:techniques}

This appendix presents a high-level catalog of 100 engineering techniques and methodological practices employed in the Strategickhaos Autonomous Cognitive-Systems Engineering (SACSE) framework. Operationally sensitive implementation details that could enable misuse are redacted and marked \textbf{[REDACTED --- OPERATIONAL SAFETY]}. Each entry includes a neutral description and one or more peer-reviewed citations.

%----------------------------------------------------------------------
% CLUSTER 1: Distributed Cognition Foundations (Techniques 1-10)
%----------------------------------------------------------------------
\subsection{Cluster 1: Distributed Cognition Foundations}

\textbf{Technique 1: Cognitive Task Analysis.}
Systematic decomposition of cognitive work into subtasks, decision points, and information flows to map human--system interaction boundaries~\cite{hutchins1995}.
\textit{Implementation:} [REDACTED --- OPERATIONAL SAFETY]

\textbf{Technique 2: Joint Cognitive Systems Design.}
Design methodology treating human and AI components as a unified cognitive system with shared goals and complementary capabilities~\cite{hollnagel2005}.
\textit{Implementation:} [REDACTED --- OPERATIONAL SAFETY]

\textbf{Technique 3: Situated Cognition Modeling.}
Contextual modeling approach that accounts for environmental and social factors influencing cognitive performance~\cite{suchman2007}.
\textit{Implementation:} [REDACTED --- OPERATIONAL SAFETY]

\textbf{Technique 4: Activity Theory Framework.}
Analytical framework examining mediated activity through tools, rules, community, and division of labor~\cite{engestrom1987}.
\textit{Implementation:} [REDACTED --- OPERATIONAL SAFETY]

\textbf{Technique 5: Naturalistic Decision Making.}
Study of expert decision-making in complex, time-pressured, uncertain environments~\cite{klein2008}.
\textit{Implementation:} [REDACTED --- OPERATIONAL SAFETY]

\textbf{Technique 6: Cognitive Load Management.}
Techniques for optimizing information presentation to minimize extraneous cognitive load while maximizing germane load~\cite{sweller2011}.
\textit{Implementation:} [REDACTED --- OPERATIONAL SAFETY]

\textbf{Technique 7: Shared Mental Model Construction.}
Methods for building aligned representations of system state and goals across human--AI team members~\cite{cannon2001}.
\textit{Implementation:} [REDACTED --- OPERATIONAL SAFETY]

\textbf{Technique 8: Ecological Interface Design.}
Interface design approach revealing system constraints and work domain structure to support adaptive behavior~\cite{vicente1999}.
\textit{Implementation:} [REDACTED --- OPERATIONAL SAFETY]

\textbf{Technique 9: Macrocognition Analysis.}
Examination of cognitive functions at the system level including sensemaking, coordination, and replanning~\cite{klein2006}.
\textit{Implementation:} [REDACTED --- OPERATIONAL SAFETY]

\textbf{Technique 10: Cognitive Work Analysis.}
Formative approach analyzing work domain constraints to inform system design~\cite{rasmussen1994}.
\textit{Implementation:} [REDACTED --- OPERATIONAL SAFETY]

%----------------------------------------------------------------------
% CLUSTER 2: Active Inference and Predictive Processing (Techniques 11-20)
%----------------------------------------------------------------------
\subsection{Cluster 2: Active Inference and Predictive Processing}

\textbf{Technique 11: Free Energy Principle Application.}
Theoretical framework for understanding self-organizing systems through variational free energy minimization~\cite{friston2010}.
\textit{Implementation:} [REDACTED --- OPERATIONAL SAFETY]

\textbf{Technique 12: Predictive Coding Architecture.}
Hierarchical generative model implementation for perception and inference~\cite{rao1999}.
\textit{Implementation:} [REDACTED --- OPERATIONAL SAFETY]

\textbf{Technique 13: Active Inference Control.}
Policy selection through expected free energy minimization incorporating epistemic and pragmatic value~\cite{friston2024}.
\textit{Implementation:} [REDACTED --- OPERATIONAL SAFETY]

\textbf{Technique 14: Bayesian Belief Updating.}
Principled uncertainty quantification and belief revision through Bayesian inference~\cite{pearl2009}.
\textit{Implementation:} [REDACTED --- OPERATIONAL SAFETY]

\textbf{Technique 15: Precision-Weighted Prediction Errors.}
Attention mechanism based on confidence-weighted prediction error signals~\cite{feldman2010}.
\textit{Implementation:} [REDACTED --- OPERATIONAL SAFETY]

\textbf{Technique 16: Generative Model Construction.}
Building probabilistic models capturing causal structure of task domains~\cite{goodfellow2016}.
\textit{Implementation:} [REDACTED --- OPERATIONAL SAFETY]

\textbf{Technique 17: Epistemic Foraging.}
Information-seeking behavior optimization for uncertainty reduction~\cite{friston2015}.
\textit{Implementation:} [REDACTED --- OPERATIONAL SAFETY]

\textbf{Technique 18: Markov Blanket Identification.}
Boundary identification for autonomous agent--environment interaction modeling~\cite{kirchhoff2018}.
\textit{Implementation:} [REDACTED --- OPERATIONAL SAFETY]

\textbf{Technique 19: Variational Message Passing.}
Efficient approximate inference in hierarchical generative models~\cite{winn2005}.
\textit{Implementation:} [REDACTED --- OPERATIONAL SAFETY]

\textbf{Technique 20: Expected Free Energy Planning.}
Action selection incorporating both information gain and goal achievement~\cite{parr2022}.
\textit{Implementation:} [REDACTED --- OPERATIONAL SAFETY]

%----------------------------------------------------------------------
% CLUSTER 3: Retrieval-Augmented Generation (Techniques 21-30)
%----------------------------------------------------------------------
\subsection{Cluster 3: Retrieval-Augmented Generation}

\textbf{Technique 21: Dense Passage Retrieval.}
Neural embedding-based document retrieval for knowledge-intensive tasks~\cite{lewis2020rag}.
\textit{Implementation:} [REDACTED --- OPERATIONAL SAFETY]

\textbf{Technique 22: Retrieval-Augmented Language Models.}
Integration of external knowledge bases with generative language models~\cite{guu2020}.
\textit{Implementation:} [REDACTED --- OPERATIONAL SAFETY]

\textbf{Technique 23: Vector Database Integration.}
Efficient similarity search infrastructure for embedding-based retrieval~\cite{johnson2019faiss}.
\textit{Implementation:} [REDACTED --- OPERATIONAL SAFETY]

\textbf{Technique 24: Chunking Strategies.}
Document segmentation approaches balancing context preservation and retrieval precision~\cite{chen2024chunking}.
\textit{Implementation:} [REDACTED --- OPERATIONAL SAFETY]

\textbf{Technique 25: Hybrid Search Architectures.}
Combination of sparse (BM25) and dense retrieval methods~\cite{karpukhin2020}.
\textit{Implementation:} [REDACTED --- OPERATIONAL SAFETY]

\textbf{Technique 26: Re-ranking Pipelines.}
Multi-stage retrieval with cross-encoder re-ranking for precision improvement~\cite{nogueira2019}.
\textit{Implementation:} [REDACTED --- OPERATIONAL SAFETY]

\textbf{Technique 27: Query Expansion.}
Automatic query augmentation for improved recall~\cite{jagerman2023}.
\textit{Implementation:} [REDACTED --- OPERATIONAL SAFETY]

\textbf{Technique 28: Context Window Optimization.}
Strategies for efficient use of limited context capacity in LLMs~\cite{liu2024context}.
\textit{Implementation:} [REDACTED --- OPERATIONAL SAFETY]

\textbf{Technique 29: Knowledge Graph Integration.}
Structured knowledge enhancement of retrieval pipelines~\cite{pan2023kgrag}.
\textit{Implementation:} [REDACTED --- OPERATIONAL SAFETY]

\textbf{Technique 30: Retrieval Feedback Loops.}
Iterative retrieval refinement based on generation quality signals~\cite{asai2023selfrag}.
\textit{Implementation:} [REDACTED --- OPERATIONAL SAFETY]

%----------------------------------------------------------------------
% CLUSTER 4: Cryptographic Provenance (Techniques 31-40)
%----------------------------------------------------------------------
\subsection{Cluster 4: Cryptographic Provenance}

\textbf{Technique 31: Content-Addressable Storage.}
Hash-based content identification ensuring integrity verification~\cite{mazieres2000}.
\textit{Implementation:} [REDACTED --- OPERATIONAL SAFETY]

\textbf{Technique 32: Merkle Tree Structures.}
Hierarchical hashing for efficient integrity verification of large datasets~\cite{merkle1980}.
\textit{Implementation:} [REDACTED --- OPERATIONAL SAFETY]

\textbf{Technique 33: Digital Signature Schemes.}
Cryptographic authentication of data origin and integrity~\cite{goldwasser1988}.
\textit{Implementation:} [REDACTED --- OPERATIONAL SAFETY]

\textbf{Technique 34: Timestamping Services.}
Trusted third-party attestation of document existence at specific times~\cite{haber1991}.
\textit{Implementation:} [REDACTED --- OPERATIONAL SAFETY]

\textbf{Technique 35: Audit Log Immutability.}
Append-only logging with cryptographic chaining for tamper evidence~\cite{schneier1999}.
\textit{Implementation:} [REDACTED --- OPERATIONAL SAFETY]

\textbf{Technique 36: Zero-Knowledge Proofs.}
Verification of computational claims without revealing sensitive inputs~\cite{goldwasser1989}.
\textit{Implementation:} [REDACTED --- OPERATIONAL SAFETY]

\textbf{Technique 37: Commitment Schemes.}
Cryptographic protocols for binding value commitment before revelation~\cite{pedersen1992}.
\textit{Implementation:} [REDACTED --- OPERATIONAL SAFETY]

\textbf{Technique 38: Verifiable Computation.}
Proof systems enabling verification of outsourced computation correctness~\cite{gennaro2010}.
\textit{Implementation:} [REDACTED --- OPERATIONAL SAFETY]

\textbf{Technique 39: Key Management Infrastructure.}
Secure generation, storage, and rotation of cryptographic keys~\cite{barker2020nist}.
\textit{Implementation:} [REDACTED --- OPERATIONAL SAFETY]

\textbf{Technique 40: Blockchain Anchoring.}
Distributed ledger integration for decentralized timestamp anchoring~\cite{nakamoto2008}.
\textit{Implementation:} [REDACTED --- OPERATIONAL SAFETY]

%----------------------------------------------------------------------
% CLUSTER 5: Reproducibility Engineering (Techniques 41-50)
%----------------------------------------------------------------------
\subsection{Cluster 5: Reproducibility Engineering}

\textbf{Technique 41: Containerized Computation.}
Reproducible runtime environments through container technology~\cite{merkel2014docker}.
\textit{Implementation:} [REDACTED --- OPERATIONAL SAFETY]

\textbf{Technique 42: Version Control Integration.}
Systematic tracking of code, configuration, and data changes~\cite{chacon2014git}.
\textit{Implementation:} [REDACTED --- OPERATIONAL SAFETY]

\textbf{Technique 43: Dependency Pinning.}
Exact specification of software dependencies for reproducible builds~\cite{stodden2023}.
\textit{Implementation:} [REDACTED --- OPERATIONAL SAFETY]

\textbf{Technique 44: Experiment Tracking.}
Systematic logging of hyperparameters, metrics, and artifacts~\cite{zaharia2018mlflow}.
\textit{Implementation:} [REDACTED --- OPERATIONAL SAFETY]

\textbf{Technique 45: Data Versioning.}
Version control for datasets enabling experiment reproducibility~\cite{iterative2020dvc}.
\textit{Implementation:} [REDACTED --- OPERATIONAL SAFETY]

\textbf{Technique 46: Literate Programming.}
Integrated documentation and executable code in reproducible notebooks~\cite{knuth1984}.
\textit{Implementation:} [REDACTED --- OPERATIONAL SAFETY]

\textbf{Technique 47: Continuous Integration Testing.}
Automated validation of code changes against test suites~\cite{fowler2006ci}.
\textit{Implementation:} [REDACTED --- OPERATIONAL SAFETY]

\textbf{Technique 48: Random Seed Management.}
Deterministic pseudorandom number generation for reproducible stochastic processes~\cite{pineau2021}.
\textit{Implementation:} [REDACTED --- OPERATIONAL SAFETY]

\textbf{Technique 49: Environment Specification.}
Declarative environment definitions for consistent deployments~\cite{nix2003}.
\textit{Implementation:} [REDACTED --- OPERATIONAL SAFETY]

\textbf{Technique 50: Artifact Archival.}
Long-term preservation of computational artifacts and data~\cite{wilkinson2016fair}.
\textit{Implementation:} [REDACTED --- OPERATIONAL SAFETY]

%----------------------------------------------------------------------
% CLUSTER 6: Security and Threat Modeling (Techniques 51-60)
%----------------------------------------------------------------------
\subsection{Cluster 6: Security and Threat Modeling}

\textbf{Technique 51: STRIDE Threat Modeling.}
Systematic identification of security threats by category~\cite{shostack2014}.
\textit{Implementation:} [REDACTED --- OPERATIONAL SAFETY]

\textbf{Technique 52: Attack Tree Analysis.}
Hierarchical decomposition of attack goals into component steps~\cite{schneier1999attack}.
\textit{Implementation:} [REDACTED --- OPERATIONAL SAFETY]

\textbf{Technique 53: Defense in Depth.}
Layered security controls providing redundant protection~\cite{nist2020csf}.
\textit{Implementation:} [REDACTED --- OPERATIONAL SAFETY]

\textbf{Technique 54: Principle of Least Privilege.}
Minimal access rights assignment for system components~\cite{saltzer1975}.
\textit{Implementation:} [REDACTED --- OPERATIONAL SAFETY]

\textbf{Technique 55: Input Validation.}
Comprehensive verification of untrusted input data~\cite{owasp2021}.
\textit{Implementation:} [REDACTED --- OPERATIONAL SAFETY]

\textbf{Technique 56: Secure Defaults.}
System configuration defaulting to restrictive security posture~\cite{mcgraw2006}.
\textit{Implementation:} [REDACTED --- OPERATIONAL SAFETY]

\textbf{Technique 57: Adversarial Testing.}
Systematic probing for security vulnerabilities through simulated attacks~\cite{mitre2023}.
\textit{Implementation:} [REDACTED --- OPERATIONAL SAFETY]

\textbf{Technique 58: Security Monitoring.}
Real-time detection of anomalous and potentially malicious activity~\cite{chandola2009}.
\textit{Implementation:} [REDACTED --- OPERATIONAL SAFETY]

\textbf{Technique 59: Incident Response Planning.}
Documented procedures for security event handling~\cite{cichonski2012}.
\textit{Implementation:} [REDACTED --- OPERATIONAL SAFETY]

\textbf{Technique 60: Secure Software Development Lifecycle.}
Integration of security practices throughout development~\cite{microsoft2023sdl}.
\textit{Implementation:} [REDACTED --- OPERATIONAL SAFETY]

%----------------------------------------------------------------------
% CLUSTER 7: Human-AI Collaboration (Techniques 61-70)
%----------------------------------------------------------------------
\subsection{Cluster 7: Human-AI Collaboration}

\textbf{Technique 61: Explanatory AI Interfaces.}
User interface patterns supporting model interpretability~\cite{gunning2019xai}.
\textit{Implementation:} [REDACTED --- OPERATIONAL SAFETY]

\textbf{Technique 62: Calibrated Confidence Display.}
Uncertainty communication aligned with actual model reliability~\cite{guo2017calibration}.
\textit{Implementation:} [REDACTED --- OPERATIONAL SAFETY]

\textbf{Technique 63: Human-in-the-Loop Workflows.}
Process designs incorporating human review at critical decision points~\cite{mosqueira2023hitl}.
\textit{Implementation:} [REDACTED --- OPERATIONAL SAFETY]

\textbf{Technique 64: Appropriate Trust Calibration.}
Interface design promoting neither over-reliance nor under-utilization~\cite{lee2004trust}.
\textit{Implementation:} [REDACTED --- OPERATIONAL SAFETY]

\textbf{Technique 65: Contrastive Explanations.}
Counterfactual reasoning for understanding model decisions~\cite{miller2019}.
\textit{Implementation:} [REDACTED --- OPERATIONAL SAFETY]

\textbf{Technique 66: Progressive Disclosure.}
Layered information presentation matching user expertise~\cite{tidwell2020}.
\textit{Implementation:} [REDACTED --- OPERATIONAL SAFETY]

\textbf{Technique 67: Interactive Model Steering.}
User controls for refining and directing model behavior~\cite{amershi2019guidelines}.
\textit{Implementation:} [REDACTED --- OPERATIONAL SAFETY]

\textbf{Technique 68: Feedback Collection Systems.}
Structured mechanisms for capturing user corrections and preferences~\cite{ouyang2022}.
\textit{Implementation:} [REDACTED --- OPERATIONAL SAFETY]

\textbf{Technique 69: Adaptive Automation Levels.}
Dynamic allocation of tasks between human and AI agents~\cite{parasuraman2000}.
\textit{Implementation:} [REDACTED --- OPERATIONAL SAFETY]

\textbf{Technique 70: Error Recovery Affordances.}
Interface patterns supporting graceful handling of AI mistakes~\cite{amershi2019guidelines}.
\textit{Implementation:} [REDACTED --- OPERATIONAL SAFETY]

%----------------------------------------------------------------------
% CLUSTER 8: Language Model Engineering (Techniques 71-80)
%----------------------------------------------------------------------
\subsection{Cluster 8: Language Model Engineering}

\textbf{Technique 71: Prompt Engineering.}
Systematic design of input prompts for optimal model performance~\cite{liu2023prompt}.
\textit{Implementation:} [REDACTED --- OPERATIONAL SAFETY]

\textbf{Technique 72: Chain-of-Thought Prompting.}
Eliciting step-by-step reasoning for complex tasks~\cite{wei2022cot}.
\textit{Implementation:} [REDACTED --- OPERATIONAL SAFETY]

\textbf{Technique 73: In-Context Learning.}
Few-shot example provision for task specification~\cite{brown2020gpt3}.
\textit{Implementation:} [REDACTED --- OPERATIONAL SAFETY]

\textbf{Technique 74: Constitutional AI Alignment.}
Self-critique and revision based on explicit principles~\cite{bai2022constitutional}.
\textit{Implementation:} [REDACTED --- OPERATIONAL SAFETY]

\textbf{Technique 75: Output Validation Pipelines.}
Systematic verification of model outputs before use~\cite{perez2022red}.
\textit{Implementation:} [REDACTED --- OPERATIONAL SAFETY]

\textbf{Technique 76: Temperature and Sampling Control.}
Generation diversity management through sampling parameters~\cite{holtzman2020curious}.
\textit{Implementation:} [REDACTED --- OPERATIONAL SAFETY]

\textbf{Technique 77: Token Probability Analysis.}
Examination of model confidence through output distributions~\cite{kadavath2022}.
\textit{Implementation:} [REDACTED --- OPERATIONAL SAFETY]

\textbf{Technique 78: Multi-Model Ensemble Methods.}
Combining outputs from multiple models for improved reliability~\cite{wang2023selfconsistency}.
\textit{Implementation:} [REDACTED --- OPERATIONAL SAFETY]

\textbf{Technique 79: Model Capability Boundaries.}
Systematic identification of task-specific model limitations~\cite{srivastava2022beyond}.
\textit{Implementation:} [REDACTED --- OPERATIONAL SAFETY]

\textbf{Technique 80: Inference Optimization.}
Techniques for efficient model deployment and execution~\cite{dettmers2022llmint8}.
\textit{Implementation:} [REDACTED --- OPERATIONAL SAFETY]

%----------------------------------------------------------------------
% CLUSTER 9: Ethical AI and Governance (Techniques 81-90)
%----------------------------------------------------------------------
\subsection{Cluster 9: Ethical AI and Governance}

\textbf{Technique 81: AI Ethics Impact Assessment.}
Systematic evaluation of potential harms and benefits~\cite{raji2020}.
\textit{Implementation:} [REDACTED --- OPERATIONAL SAFETY]

\textbf{Technique 82: Bias Detection and Mitigation.}
Methods for identifying and reducing unfair model behavior~\cite{mehrabi2021}.
\textit{Implementation:} [REDACTED --- OPERATIONAL SAFETY]

\textbf{Technique 83: Privacy-Preserving Techniques.}
Data protection methods including differential privacy~\cite{dwork2014}.
\textit{Implementation:} [REDACTED --- OPERATIONAL SAFETY]

\textbf{Technique 84: Model Cards and Documentation.}
Standardized model capability and limitation documentation~\cite{mitchell2019}.
\textit{Implementation:} [REDACTED --- OPERATIONAL SAFETY]

\textbf{Technique 85: Datasheets for Datasets.}
Comprehensive dataset documentation practices~\cite{gebru2021}.
\textit{Implementation:} [REDACTED --- OPERATIONAL SAFETY]

\textbf{Technique 86: Stakeholder Engagement.}
Inclusive processes for affected party input~\cite{sloane2022}.
\textit{Implementation:} [REDACTED --- OPERATIONAL SAFETY]

\textbf{Technique 87: Audit Trail Requirements.}
Documentation standards for accountability~\cite{kroll2017}.
\textit{Implementation:} [REDACTED --- OPERATIONAL SAFETY]

\textbf{Technique 88: Algorithmic Impact Statements.}
Public documentation of AI system effects~\cite{metcalf2021}.
\textit{Implementation:} [REDACTED --- OPERATIONAL SAFETY]

\textbf{Technique 89: Red Team Evaluations.}
Adversarial testing for safety and alignment~\cite{ganguli2022}.
\textit{Implementation:} [REDACTED --- OPERATIONAL SAFETY]

\textbf{Technique 90: Responsible Disclosure Policies.}
Protocols for communicating discovered vulnerabilities~\cite{christey2002}.
\textit{Implementation:} [REDACTED --- OPERATIONAL SAFETY]

%----------------------------------------------------------------------
% CLUSTER 10: Advanced and Emerging Techniques (Techniques 91-100)
%----------------------------------------------------------------------
\subsection{Cluster 10: Advanced and Emerging Techniques}

\textbf{Technique 91: Multi-Agent Coordination.}
Orchestration of multiple AI agents for complex task completion~\cite{wu2023autogen}.
\textit{Implementation:} [REDACTED --- OPERATIONAL SAFETY]

\textbf{Technique 92: Tool-Use Augmentation.}
External tool integration for extending model capabilities~\cite{schick2023toolformer}.
\textit{Implementation:} [REDACTED --- OPERATIONAL SAFETY]

\textbf{Technique 93: Long-Context Processing.}
Techniques for handling extended context windows~\cite{chen2023longlora}.
\textit{Implementation:} [REDACTED --- OPERATIONAL SAFETY]

\textbf{Technique 94: Multimodal Integration.}
Unified processing of text, image, and other modalities~\cite{openai2023gpt4}.
\textit{Implementation:} [REDACTED --- OPERATIONAL SAFETY]

\textbf{Technique 95: Reinforcement Learning from Human Feedback.}
Alignment through preference-based reward modeling~\cite{christiano2017}.
\textit{Implementation:} [REDACTED --- OPERATIONAL SAFETY]

\textbf{Technique 96: Continual Learning Approaches.}
Methods for model adaptation without catastrophic forgetting~\cite{kirkpatrick2017}.
\textit{Implementation:} [REDACTED --- OPERATIONAL SAFETY]

\textbf{Technique 97: Federated Learning Integration.}
Privacy-preserving distributed model training~\cite{mcmahan2017}.
\textit{Implementation:} [REDACTED --- OPERATIONAL SAFETY]

\textbf{Technique 98: Neuro-Symbolic Hybrid Methods.}
Integration of neural and symbolic reasoning approaches~\cite{garcez2020}.
\textit{Implementation:} [REDACTED --- OPERATIONAL SAFETY]

\textbf{Technique 99: Interpretability Engineering.}
Systematic development of model explanation capabilities~\cite{doshi2017}.
\textit{Implementation:} [REDACTED --- OPERATIONAL SAFETY]

\textbf{Technique 100: Sovereign Architecture Patterns.}
Design patterns for autonomous, self-governing cognitive systems~\cite{hutchins1995}.
\textit{Implementation:} [REDACTED --- OPERATIONAL SAFETY]

%----------------------------------------------------------------------
% IEEE Keywords
%----------------------------------------------------------------------
\begin{IEEEkeywords}
Distributed cognition, active inference, cryptographic provenance, reproducibility engineering, human--AI co-creation, responsible redaction
\end{IEEEkeywords}

\end{appendices}

%----------------------------------------------------------------------
% Bibliography
%----------------------------------------------------------------------
\bibliographystyle{IEEEtran}
\bibliography{references}

\end{document}
